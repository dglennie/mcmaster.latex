\section{Summary \& Thesis Conclusions}
Skin redness (and its related increase in the total hemoglobin concentration) is an important indicator of patient response to a variety of treatments due to its relation to inflammation and tissue oxygenation. Aside from visual assessment which is subjective and, therefore unreliable, diffuse reflectance spectroscopy (DRS) is arguably the most popular approach to monitoring changes in the hemoglobin concentration in skin. Regardless of the measurement approach and subsequent analysis, previous attempts were only capable of accurately measuring hemoglobin concentration changes in the absence of other chromophore concentration changes. In addition, all of the systems were difficult to use (required extensive training or precise measurement) and/or very expensive. The papers of this thesis present an integrating sphere-based total DRS system and spectrally-constrained model that is not only capable of accounting for concentration changes to non-hemoglobin chromophores, but is also user-friendly and cost effective.

The majority of DRS systems rely on fiber-based reflectance spectroscopy that requires training and an understanding of optics to be correctly used. In contrast, an integrating sphere-based system requires little to no formal training prior to use. However, integrating spheres are rarely used because their measured reflectance spectra are not easily converted into the true reflectance spectra, which is a necessary step if spectrally-constrained analysis is to be performed. Before an integrating sphere-based system can be used in this way, its response must be thoroughly characterized. The system presented in Chapter~\ref{chap:p1-system} was tested for wavelength accuracy, light source stability, measurement uncertainty and reproducibility. While the individual measurement uncertainty was intensity dependent, it was found to never exceed 1\% across the 500 to 700 nm spectrum. The steps required to correct for the Single Beam Substitution Error (SBSE) were outlined and, once performed, resulted in an accurate recovery of the reflectance spectra (to within the measurement uncertainty) for a set of four colored calibration standards.

The hypothesis that a cheaper, more user-friendly system would facilitate clinical research was strengthened when the system outlined in Chapter~\ref{chap:p1-system} was immediately used to determine the time to maximal effect of epinephrine (a vasoconstrictor). Prior to the work presented in Chapter~\ref{chap:p2-mckee}, the commonly cited time interval between injection and incision was 7 to 10 minutes, as measured in 1987 by Larrabee et al.\cite{Larrabee1987} using laser Doppler flowmetry. In this study, 12 volunteers were injected either with lidocaine (a vasodilator) or lidocaine with epinephrine. DRS measurements were performed with this system over a two hour time period. Since work on a spectrally-constrained model was still being undertaken at the time of this study, the Dawson Erythema Index (EI)\cite{Dawson1980} was used to measure changes in the hemoglobin concentration in the skin. The results showed that tiem time required to reach the lowest hemoglobin concentration was 25.9 minutes $\pm$ 5.1 minutes (95CI) which was significantly longer than the previously stated time. The importance of this research is illustrated by the fact that, within the first year of publication, this paper was cited 20 times.

The spectrally-constrained model developed for use with the integrating sphere-based system (Chapter~\ref{chap:p3-model}) was based on the diffusion theory equation by Farrell et al.\cite{Farrell1992} In addition to the SBSE, another correction factor was required to account for scattering losses at the detection port of the integrating sphere. The functional form of this new correction factor was determined with Monte Carlo simulations and confirmed with tissue-simulating liquid phantoms. A fitting algorithm was devised that applied the different corrections at strategic positions along the chain, depending on the information required to compute them. The parameters for the model were the concentrations for the individual chromophores found in the epidermal and dermal layers of human skin. Since the reduced scattering coefficient spectrum of human skin has been well established and varies minimally between healthy individuals, an assumed scattering spectrum was used in the model. For this reason, as well as because of the layered structure of skin, absolute hemoglobin concentrations were not possible. However, relative increases and decreases in the total hemoglobin concentration were possible. Unlike absorbance-based analysis approaches, this spectrally-constrained model was able to account for changes in the concentrations of the other skin chromophores. The model was tested on the spectra obtained during the epinephrine study (Chapter~\ref{chap:p2-mckee}) and was shown to agree with the Dawson EI calculations except immediately following injection. Since the Dawson approach is an absorbance-based model and is not capable of accounting for other concentration changes, it was hypothesized that the difference between the two analysis methods was due to an increase in bleeding or swelling that led to an underestimation in the true hemoglobin concentration with the Dawson approach. 

Once the system and model were fully tested, they were used in a study to measure erythema in cancer patients undergoing intensity modulated radiation therapy (IMRT) (see Chapter~\ref{chap:p4-imrt_study}). DRS measurements were performed prior to each delivered fraction of radiation. In addition, weekly visual assessments (VA) by the attending oncologist and radiation dosimetry measurements with TLDs were performed. Due to limitations in the study design, a direct comparison between VA and DRS was not possible, but conclusions regarding the optimum use of the DRS system were clearly illustrated. Since DRS measurements are quantitative, they are able to detect the small daily variations in the hemoglobin concentration in human skin that are not visible to the human eye. From control measurements performed outside of the radiation field, the daily variation in hemoglobin concentration was found to be 15.6\% (st.dev.). For this reason, daily measurements are recommended along with a smoothing filter to improve the interpretation of the results. Using the daily control measurements to determine a minimum detectable increase (MDI) threshold, the DRS approach was capable of identifying skin redness anywhere from 1 to 19 days prior to visual detection. The guidelines in this paper will provide researchers with a standard protocol for using a DRS system to monitor skin redness in studies that compare treatment or skin care regimens.

Since TLD measurements were performed once a week on each patient, there was a sufficient amount of data to perform descriptive statistics to determine the approximate uncertainty added to TLD measurements due to patient and TLD positioning errors (see Chapter~\ref{chap:p5-tlds}). Once the data was confirmed to have no temporal relationship, statistical analysis showed that the range of the percent standard deviations (PSDs) was 29.0\%. The distribution of the PSDs lead to an approximation of the population standard deviation of 9.1\%, only 6\% greater than the standard uncertainty in TLDs (commonly cited as 2-3\%). Limited investigations into the causes of these uncertainties showed a statistically significant relationship between the PSDs and their corresponding surface dose gradients.

\section{Future Work}
This thesis presented a novel approach to measuring changes in the hemoglobin concentration in skin. Its development and use, as presented in the papers of this thesis, raised many questions that would be ideal topics for further research. For example, the system was built with components that were on-hand at the time of construction. This included an older laptop, light source, and spectrometer. All further research should be undertaken with an updated system with newer components.

The integrating sphere size was limited by the dimensions of a block of Spectralon\textregistered~that had been previously purchased. The final sphere geometry was based on a section of integrating sphere theory that indicates, to a limit, that larger integrating spheres (and, as a consequence, the smaller port fractions) make for better illumination and detection devices. This assumption can and should be tested with Monte Carlo simulations to determine how small of a sphere is still useful in this particular application. It may be that small spheres would fit inside the human mouth, expanding the applications of this to oral mucositis, another common side effect of head and neck IMRT.\cite{Hancock2003}

Other aspects of the system can also be modified to expand the number of applications. For example, by changing the wavelength range and modifying the model, this system and model can be used to monitor the bilirubin concentrations in the skin of jaundiced neonates.\cite{Randeberg2005} There is also a need in forensic medicine for a quantitative method of dating bruises,\cite{Randeberg2006} the coloring of which is due to the breakdown of hemoglobin into waste products such as bilirubin. Finally, there are also several possibilities relating to breast cancer in both tumour detection\cite{Peters1990,Tromberg2005} and the monitoring of patient response.\cite{Porock1999,Tromberg2010}

When characterizing the system response to convert the detected reflectance spectra into the true reflectance spectra, a scattering losses correction factor (SLCF) was necessary to account for an increase in the fraction of photons that scattered away from the detection port between when the Spectralon\textregistered~standard and the actual sample were measured. With recent advances in tissue simulating phantoms, it may be possible to create a solid phantom with similar scattering properties as human skin. If this sample was used as the calibration plate, it may eliminate the need for the SLCF, thereby streamlining the system characterization process.

One of the limitations of the erythema study was that the protocol did not include a purposeful daily visual assessment of the erythema region. For this measurement approach to be more readily embraced, it must be shown to be concretely superior to visual assessment. Therefore, another study (possibly with additional objectives such as regimen comparison) should include both daily assessments (visual and DRS). While this may seem time-consuming, it is a necessary step if this system is ever to become the gold standard for skin reaction comparison and monitoring. The visual assessments do not need to be performed by the attending oncologist (as they were in Chapter~\ref{chap:p4-imrt_study}), but the person must be properly trained in the interpretation of the visual assessment scale.

The use of the system and model for both the erythema (Chapter~\ref{chap:p4-imrt_study} and epinephrine (Chapter~\ref{chap:p2-mckee}) studies raised important questions about the minimum increase in the total hemoglobin in skin required for visual identification. Both papers suggested that this threshold was significantly higher than originally thought by the investigators. Therefore, a study should be conducted in which redness is induced to varying degrees in different patches of skin, all in the same region on a single volunteer. Assessment would be done both visually and with the DRS system. Redness could either be induced with varying concentrations of lidocaine (either injected or topically applied), or with different doses of ultraviolet radiation.\cite{Diffey1991,Harrison2002} In addition, volunteers for this study should span the Fitzpatrick scale,\cite{Fitzpatrick1988} as redness will be harder to visually detect when more melanin is present in the skin.

There are many potential uses and applications of this DRS system and analysis model. It is the hope of this author that the work included in this thesis provides a clear base for future research into improving the system and analysis method, and that it creates opportunities for clinical research across the health care spectrum.



