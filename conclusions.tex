\section{Summary \& Thesis Conclusions}
Paragraph1 - what I researched and what original contribution to field is (DRS with IS combined with spec constrained model for HbO in skin)

Paragraph2 - main finding 1 (how arrived at finding and how challenges previous research) system

P3 - model

p4 - cheaper system = more use (illustrated by immediate use for epinehphrine)

p5 - segway into application in RT, better than VA, needs to be userfriendly, accessible (this is why I set out use guidelines)

summary statement

The hypothesis that a cheaper, more user-friendly system would facilitate clinical research was strengthened when this was immediately used to determine the time to maximal effect of epinephrine.

\section{Future Work}
This thesis presents a novel approach to measuring changes in the hemoglobin concentration in skin. Its development and use, as presented in the papers of this thesis, raised many questions that would be ideal topics for further research.

To begin, the system was built with components that were on-hand at the time of construction. This included an older laptop, light source, and spectrometer. All further research should be undertaken with an updated system with newer components. The integrating sphere size was based on the block of Spectralon\textregistered~cube currently in possession, based on a part of integrating sphere theory that indicates, to a limit, that larger integrating spheres (are as a result smaller port fractions) make for better illumination and detection geometries. This assumption can and should be tested with Monte Carlo simulations to determine how small of a sphere is still useful in this particular application. It may be that small spheres would fit inside the human mouth, expanding the applications of this to oral mucositis, another side effect of head and neck IMRT.\cite{Hancock2003} Other aspects of the system can be modified to expand the number of applications. For example, by changing the wavelength range and modifying the model, this approach can be used to monitor bilirubin concentrations in skin that are indicative of jaundice in neonates.\cite{Randeberg2005} There is also a need in forensic medicine for a quantitative method of dating bruises,\cite{Randeberg2006} the coloring of which is due to the breakdown of hemoglobin into waste products such as bilirubin. There is also several possibilities relating to breast cancer in both tumour detection\cite{Peters1990,Tromberg2005} and the monitoring of patient response.\cite{Porock1999,Tromberg2010}

When characterizing the system response to convert the detected reflectance spectra into the true reflectance spectra, a scattering losses correction factor (SLCF) was necessary to account for an increase in photons that scattered away from the detection port from when the Spectralon\textregistered~standard and the actual sample were measured. With recent advances in tissue simulating phantoms, it may be possible to create a solid phantom with similar scattering properties as human skin. If this sample was used as the calibration plate, it may eliminate the need for the SLCF, thereby streamlining the system characterization process.

One of the limitations of the erythema study was that the protocol did not include a purposeful daily visual assessment of the erythema region. For this measurement approach to be more readily embraced, it must be shown to be concretely superior to visual assessment. Therefore, another study (possibly with additional objectives such as regimen comparison) should include both daily assessments (visual and DRS). While this may seem time-consuming, it is a necessary step if this system is ever to become the gold standard for skin reaction comparison and monitoring. The visual assessments do not need to be performed by the attending oncologist (as they were in Paper~4), but the person should be properly trained in the interpretation of the visual assessment scale.

The applications of the system and model for both the IMRT patients (Paper~4) and the epinephrine study (Paper~2) raised important questions about the minimum increase in total hemoglobin in skin before visual identification of skin redness. Both papers suggested that this threshold was significantly higher than originally thought by the investigators. Therefore, a study should be conducted in which redness is induced to varying degrees in different patches of skin in the same region on a single volunteer, and then assessed visually as well as with the DRS system. This could either be achieved with varying concentrations of lidocaine, or with different doses of ultraviolet radiation.\cite{Diffey1991,Harrison2002} In addition, volunteers for this study should span the Fitzpatrick scale,\cite{Fitzpatrick1988} as redness will be harder to visually detect when more melanin is present in the skin.

There are many potential uses and applications of this DRS system and analysis model. It is the hope of this author that the work included in this thesis provides a clear base for future research into improving the system and analysis method, and that it creates opportunities for clinical research across the health care spectrum.



