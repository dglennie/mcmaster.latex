\section*{Preamble}
Following the full characterization of the measurement system (Paper~1) and the spectrally-constrained model (Paper~3), a study was devised to employ both in the clinical setting. In radiation therapy, it is common to compare two treatment or skin care management regimens using only visual assessment (VA) and patient questionnaires. Both of these evaluation techniques are subjective in nature, meaning that there is a large amount of inter- and intra- observational variation. The spectrally-constrained integrating sphere-based diffuse reflectance spectroscopy approach outline in the previous papers provides an objective way to quantify patient skin reactions, but is not as easily implemented as VA.

This paper explains where the difficulty in implementation comes from, and how it can be best addressed. This is a very important step in making the system and accompanying model accessible to the clinical community. More clinicians will adopt this assessment method if they have a set of recommendations on how it should be used to optimize clarity in the results.

This study required research ethics board (REB) approval. The application was prepared by the author of this thesis (further known as ``the author'') with significant assistance from Mrs. L. Doerwald-Munoz and under the guidance of Drs. T. Farrell, J. Hayward, O. Ostapiak, and J. Wright. The daily measurements were very time-consuming and required much assistance from Mrs. Doerwald-Munoz with occasional help from Miss. C. Cook and Dr. P. Muruganandam. All analysis was performed by the author. The manuscript was written by the author under the guidance of Drs. Farrell and Hayward and was additionally editted by the other authors, Drs. Ostapiak and Wright, and Mrs. Doerwald-Munoz. The manuscript has been altered from its original form to match the style of this thesis.

\section*{Contents}

\begin{center}
	
	\textbf{Diffuse reflectance spectroscopy for monitoring erythema in head \& neck intensity modulated radiation therapy}
	
	Diana L. Glennie, Joseph E. Hayward, Lillian Doerwald-Munoz, James Wright, and Thomas J. Farrell
	
	\textit{Department of Medical Physics and Applied Radiation Sciences, McMaster University, 1280 Main Street West, Hamilton, Ontario, L8S 1A8}
	
	\textit{AND}
	
	\textit{Department of Medical Physics, Juravinski Cancer Centre, 699 Concession Street, Hamilton, Ontario, L8V 5C2}
	
\end{center}

\noindent Submitted to the \textit{Journal of Radiation Oncology} on DATE.

\section*{Abstract}
Text

\section{Introduction}
Text